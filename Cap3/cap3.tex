%In this chapter we summarize existing work related to this research.

Path planning for autonomous UAVs is a current research topic, with experiments being presented in \cite{chen_2016, silva_2017, radmanesh_2016, lu_2017}. To the best of our knowledge, research on UAVs exchanging real-time critical information and adopting approaches to improve their stealthiness on hostile environments by using different flight formations has not been deeply addressed in the literature. In this direction, this chapter presents some works that are somehow related to our research, especially in stealth aspects.

Literature presents several approaches for planning UAVs paths, such as approaches based on genetic algorithm \cite{he_2013}, neural networks \cite{glasius_1995}, and heuristics, such as the TangentBug \cite{breitenmoser2010}. A widely used technique for path planing with obstacle avoidance \cite{wzorek_2006, kothari_2009, zhen_2014, wen_2017, lu_2017} is the Rapidly-Exploring Random Tree (RRT) \cite{lavalle_1998}, which is used in this work.

Strongly related to this research on the topic of stealth is the work related in \cite{turgut_2009}, which presents a collaborative approach to improve stealth dissemination by sharing information between nodes in a sensor network. Stealth dissemination means to reduce the exposure of the data shared by nodes of the network. In this work, the authors propose a method to quantify the stealth level of a sensor node and introduce a communication protocol based on the try-and-bounce approach. In this protocol, each node calculates its level of stealth that, if below a threshold, drives the node to not respond to messages sent by another nodes. The absence of a reply means that the sensor is within a threat region. We follow this approach in our model, when an UAV detects a threat and estimates its position and radius, it informs its direct neighbors about the threat detected and shuts down its communication radio, losing connectivity with all its neighbors. Consequently, neighborhood UAVs consider that this UAV is within a threat region, which should be avoided.

He and Dai \cite{he_2013} present a niche genetic algorithm\footnote{Niched genetic algorithm is an improved standard genetic algorithm, which is inspired by the niche phenomena of natural ecosystems. It not only retains the original advantages of GA, but also maintains the population diversity to solve multi-objective optimization problems \cite{zhou_1999}} for 3--dimensional stealth path planning for multiple UAVs. The authors consider that obstacles and threats vary dynamically. The weather conditions, except for wind, are neglected. The path constraints include: stealth, weather conditions, terrain avoidance, obstacles avoidance, minimum distance between UAVs in order to maintain connectivity, minimum length of path segment, limited route distance and maneuverability constraints. The stealth approach used in this work consists in optimizing the yaw and pitch angles to reduce the exposure of UAV to enemies radars. Instead of modelling the trajectory of UAVs as an optimization problem, we adopt an approach based on ordinary differential equations, as described in \cite{ghedini_2016_dars}. In addition, while the work mentioned above aims at reducing only the active signature of the UAV, our work also reduces the passive signature of the UAV by shutting down the communication radio when exposed to a threat. Moreover, we consider that the UAVs can flight using different approaches, such as trail formation or using an unstructured flight that allows mechanisms for increasing the robustness level to failures and the coverage area rate of the network.

While this work focuses on UAVs, stealth is a key component in several games, which allow the player to approach certain situations stealthily in order to achieve its goal without triggering any alarms \cite{mendonca_2015}. Mendon\c{c}a \textit{et al.} \cite{mendonca_2015} present a method to find a covert path in a terrain patrolled by multiple moving agents using a special navigation mesh, a collection of convex polygons that define which areas of an environment are traversable by agents. The generated path passes through cover whenever possible in order to avoid open areas and reduce its overall visibility. A* was the algorithm used for path planning \cite{hart_1968}. The authors implemented a policy to change the agent's speed according to its exposure to threats: the speed is reduced while the agent is in cover or near enemy patrols, and increase in open areas and away from patrolling agents. The work focus on  games where the terrain specifications are usually known to the agent.

Bellingham \textit{et al. } \cite{bellingham_2002} describes a cooperative path planning for a fleet of UAVs. The paths are model uncertainty in the environment by defining the probability of UAV loss. In order to improve the UAVs' survival probability, the authors exploited the coupling effects of cooperation between UAVs. Their algorithm uses straight-line paths to estimate the time-of-flight and risk of each mission. The missions considered in this work were scheduled in a way that one group of UAVs opens a corridor through anti-aircraft defenses before a follow-on group attacks higher value targets, with increased survival probability. This work consider that each UAV has an specific role in the network, while our model can be used on heterogeneous networks, since the UAVs have no specific role in the network. Moreover, the threats are dynamically detected in our scenario. The stealth policy to reduce the passive signature of the UAV by shutting down the communication radio when exposed to some kind of threat was included in our model different flight approaches mechanisms.


Connectivity maintenance of multi-agent system is a topic that has been well studied. It is essential for cooperative networks, because without  communication, agents cannot perform collaborative tasks. Examples of works in this field include \cite{ji_2007, dimarogonas_2010, morbidi_2010}. Our model follows the approach presented in Sabattini \textit{et al.} \cite{sabattiniijrr2013}, where the overall connectivity is maintained through a decentralized estimate of the algebraic connectivity. Moreover, this connectivity maintenance framework can be enhanced to consider additional objectives. In particular, as shown in \cite{Lee13tmech}, the concept of generalized connectivity can be used to simultaneously guarantee connectivity maintenance and collision avoidance among the UAVs.

Regarding multi-robot systems, Ghedini \textit{et al.} \cite{ghedini_2016_dars} addresses the problem of topology control to deal with node failures in networks composed by multiple robots. The robots take actions to improve the robustness when necessary. In addition, this approach is combined with a connectivity maintenance control law, thus providing a mechanism that ensures, in the absence of failures, network connectivity and an improvement in the overall robustness to failures. Furthermore, this work was extended to support a control law to improve the robot's coverage area \cite{ghedini_2017}. This approach to increase the coverage area and robustness to network failures and the way that it this is combined with the control law for connectivity maintenance is used in our work. We extended this work by including stealth.

%In our scenario there are enemy radars, which should be detected and avoided, making it necessary to add a control law responsible for defining the UAVs' trajectory. In addition, to the detriment of failures, in our case the UAVs turn off the communication radio for a certain period when exposed to a threat. \textcolor{red}{esse ultimo paragrafo parece redundante com tudo o que foi escrito ateh aqui...}
