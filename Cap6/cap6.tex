
Modelling UAVs to work collaboratively to dynamically take decisions to improve the network stealthiness is still an open issue. In this direction,  we presented a model to improve the stealthiness of UAVs in hostile environments by exploiting collaboration mechanisms for threat region detection and avoidance with a stealth policy to reduce the passive signature of the UAVs. This policy consists in turning off the UAVs communication radar when exposed to the threats to avoid passive detection. The model relies on a combined control law that handles connectivity maintenance and collision avoidance between UAVs, flight path planning, coverage area rate and robustness to failures improvement.

With regard to strategies of flight, we adopt two approaches: trail formation, which is intended to maximally reduce the exposure of UAVs to threats, and the unstructured flight approach, which can also be used for improving the coverage area or the robustness to failures while still improving stealth. In order to flying according to these approaches, we described a procedure, based on the RRT algorithm, a widely used algorithm for path planning for UAVs, for reaching the goal while avoiding threat regions. RRT was extended to support a movement constraint, based on angle validation, for avoiding movements that are impossible to be carried out by an UAV. Moreover, a procedure for smoothing the path based on horizontal highway segmentation was proposed.

The approach was validated in different simulation settings, varying the number of UAVs to 5, 10 and 30. In addition, considering that a mission can be performed in scenarios with different numbers of radars, we consider three hostility level for the scenarios: low, medium and high. Firstly, we demonstrate the importance of the stealth approach on the trail formation. The stealth level significantly increases when the stealth procedure is active, which is crucial for some military missions, such as Direct Action. The next experiment consisted in showing the importance of both the stealth approach and the connectivity maintenance control law for the unstructured flight setting. As expected, the stealth level of the network increases when the stealth approach is active, especially when the connectivity maintenance control law is active, thus highlighting the importance of collaboration among the UAVs. In addition, by using the connectivity maintenance control law, as expected, the algebraic connectivity improves too.

Once the relevance of the stealth approach and the connectivity maintenance control law were demonstrated, the last experiment was intended to evaluate the impact of the coverage area and the robustness to failure improvement control laws. The results shown that, by using the stealth approach, regardless the combination of control laws used, the stealth level of the network increases, compared to simulations performed without this approach. Besides, by using the coverage area improvement, the stealth level of the network improves, especially when combined with the robustness improvement control law. Regarding the robustness to failures improvement control law,  the robustness level of the network increases when this control law is active for low size of networks or for large sizes of networks performing missions on scenarios with low level of hostility. This occurs due to the fact that the robustness improvement reduces the coverage area rate, then when an UAV is detected by a threat, its direct neighbours are commonly detected too, due to the movement constraint based on angle variation. For medium and high level of hostility, combining the control laws for coverage area and robustness to failures improvement provides higher values of robustness level, considering large size of networks.

Considering the applicability of the model in real environments, the same can be applied in different scenarios. It is possible to perform offensive missions, which should increase the chances of some UAV to reach the goal position without being discovered by an enemy radar, thereby decreasing the chance of being shot down. In addition, by defining a set of waypoints, it is possible to conduct monitoring missions so that UAVs pass through these regions covering the largest area while attempting to reduce their exposure to the threats by using the stealth approach. The choice of flight approach is directly related to the purpose of the mission and the hostility level of the environment. Depending on the mission, a dynamic approach can also be used to vary flight formation.

In order to perform the mission in a collaborative way, this model requires only that all UAVs know the mission's goal position and are equipped with a passive receiver capable of detecting enemy radars, such as the RWR, and a radar responsible for the communication between the UAVs. In the experiments, we assume that the UAVs know no threat a priori, however this information can be passed on to the UAVs in advance.

This work did not compare the results obtained with related works, nor did it perform in-depth statistical tests to validate the adaptive gains used in the combined control law model. Therefore, these can be addressed in future research. In addition, the metric to assess the stealth level adopted in this thesis is fairly basic and can be improved in future work.

In this work, we considered a bidimensional scenario for modelling the UAV network, since our focus is to evaluate the effect of different flight approaches and control laws for the stealthiness of the network. Considering that real-world UAV applications are performed on tridimensional environments, modelling a UAV network as a bidimensional scenario makes a practical application unfeasible and is a limiting factor for stealthiness. For instance, stealthiness improvement by optimizing pitch angle is only possible in 3D scenarios. In this direction, addressing the problem of modelling a UAV network on a 3D scenario is a suggestion for future work.

Likewise, the proposed model has limitations to be implemented in real UAVs, as aspects of the UAV, such as the yaw and pitch angles variations are not considered. Moreover, the proposed model abstracts physical aspects of the communication radar and RWR. However, within what it was proposed to do, which was to verify the impact of the insertion of a stealth policy for different flight approaches in a collaborative UAV network, the results were quite satisfactory. The stealth level increased considerably when the stealth policy was active, for all flight approaches. In addition, this type of study was very little addressed in the literature, thus, to the best of our knowledge this is a pioneering work.

We assume that the UAVs can always estimate correctly the electronic position of the threats by passive detection, which is something that not occurs all the time in real application. Thus, evaluating the impact of the errors on the threat position estimation to the model can be exploited in future researches.

The proposed model does not consider the maximum operation time of the UAVs to perform a mission. Therefore, the path calculated by the UAVs may be impractical for an real application, given its length. In this context, another extension for this work can be modelling the maximum operation time of the UAVs during the computation of the path planning.

Finally, a model that optimizes the adaptive gains of the control law in order to maximize some specific purpose, being that coverage area or robustness to failure improvement can be exploited on future work.


