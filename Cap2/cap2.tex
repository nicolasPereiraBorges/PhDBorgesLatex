\section{Military Operations}

In 2011, the US special forces who raided the safe house of Osama Bin Laden caught him completely by surprise by using drones to evade the Pakistani radar [5]

The way military missions performed by aircraft are carried out continuously changes. Nowadays, environments are being monitored by different types of electromagnetic devices, making necessary to adopt new approaches to take advantage from the opponent by using the surprise factor \cite{exercito_2009, kacena1995}. 

In addition to radars, there are other kinds of treats, whose purpose is to destroy an aircraft, such as Fire-Control Radars (FCR). According to Berglund \cite{berglund_2001}, the problem of penetrating enemy air defences is a concern for most sorts of aerial vehicles. Long range strike missiles are likely to be used against the enemy air defences in conjunction with other components, such as EW. The traditional approach for defence penetration is to attempt to avoid the detection of an aircraft until it is too late for the defences to react effectively.

Considering military operations performed in hostile environments, according to \citeauthor{aeronautica_2012} \cite{aeronautica_2012}, some of missions performed by aircraft are:
\begin{itemize}
    \item \textbf{Direct Action:} to neutralize enemies for strategic or operational purposes. This action is performed on hostile environment or area controlled by the enemies. 
    \item \textbf{Approximate Air Support:} to detect, identify and neutralize enemy surface forces that are in direct contact with friendly surface forces.
    \item \textbf{Air Reconnaissance:} to gather specific data onto enemy forces and areas of interest.
    \item \textbf{Scan:} to detect, identify and neutralize enemy forces in areas of interest.
    \item \textbf{Combat Service Support:}  to provide supply, maintenance, transportation, health services, and other services required by the soldiers of combat units to continue their missions in combat
\end{itemize}

The focus of this work relies on the Combat Service Support missions. This kind of missions must address highly uncertain conditions. While perfect forecasts are rarely possible, forecast models can reduce uncertainty about what supplies or services will be needed, where and when they will be needed, or the best way to provide them. In this context, the use of stealth approaches becomes essential. The expression stealth can be defined as a combination of techniques and technologies that aims at increasing the enemy's difficulty in detecting, tracking, guiding or predicting an object's future position in space by reducing the aircraft signature \cite{kacena1995}. Singh \cite{singh_2015} describes some of these factor, as follows:

\begin{itemize}
    \item Acoustic: The acoustic signatures of an aircraft are due to the aerodynamic noise from its vertices, wings, rotors, propellers and engines. The intensity of noise is proportional to the wingspan loading and speed. Reduction of such signature by coating the engine with sound-absorptive materials contributes to acoustic stealth.
    \item Optical: The optical signature of an aircraft is related to its size, shape and contrast with the background. The background luminance also depends upon the atmospheric conditions and the target position with respect to the sun. The surface-texture of the aircraft is other factor that impacts the optical signature.
    \item Thermal: The thermal signatures are due to the heat generated by the aircraft engines jets, propellers and rotors. The exhaust heat of the aircraft can be prevented by travelling towards the ground. Moreover, low-emissive materials can be used to avoid radiation. 
    \item Radar: The radar signatures are related to the radio frequency emissions from the aircraft. They can be reduced by either applying radar absorbent material coatings or shaping the aircraft. 
\end{itemize}

%The aircraft signature can be categorized as active or passive \cite{westra_2009}. According to \citeauthor{westra_2009} \cite{westra_2009}, an active signature consists in all the observable emissions from a stealth platform. On the other hand, the passive signatures can be defined as all the observable emissions from a stealth platform that require external illumination. In terms of nomenclature, the active signature reduction methods are commonly called low probability of interception (LPI), while passive signature reduction techniques are often called low observable (LO) \cite{westra_2009, lynch_2004}. 

In terms of techniques, it is possible to reduce the airplane signature with simple approaches, such as using natural hiding places provided by the environment, such as clouds and darkness \cite{haffa_2002}. In addition, radio or radar signals transmitted by an aircraft can also be used to detect an aircraft's position, therefore, techniques to decrease the probability of intercepting those signals, such as flying with radars turned off, using electronic pulses that do not exceed the range of the threat, and using frequency-hopping radios and radars to deny electromagnetic detection are all possible stealth mechanisms\cite{haffa_2002, friedman_2010}.

Aircrafts can be equipped with different sensors and radars. Regarding stealth, an active approach to detect a threat presence can expose aircrafts to passive enemy radars. Moreover, it is highly likely that an aircraft radar will have a detection range larger than an enemy vigilance radar, therefore it would be detected in advance. Due to this situation, the detection of the threats occurs passively on hostile environments. One equipment applied for this purpose is the Radar Warning Receiver (RWR), designed to detect radars as part of the process of protecting the target against those radars. It must detect a wide range of radar signals, which can come from any direction \cite{adamy2004}. The RWR has been used successfully by modern combat aircraft, such as the F-16, for many years \cite{brownlow_2008}. Regarding UAVs, tests using RWR on Predator B/MQ-9 Reaper Block 5 were successfully conducted on April 2017, allowing UAVs to operate in the proximity of threat radars and enemy air defenses \cite{osborn_2017}. 
  
  
During the last decades, manned airplanes were used for military missions. However, the number of applications relying on UAVs has significantly increased, since they are capable of carrying out missions that are too dangerous or difficult for humans \cite{mod_2011}. According to Jentsch \cite{jentsch_2016}, on future battle spaces each service and each ally will have its own set of aerial and ground uninhabited system with increasing levels of machine intelligence. In terms of real-world applications, to augment the capabilities of ground and air forces the U.S. military have developed several UAVs to work in conjunction with human pilots, thus enhancing their surveillance and combat capabilities \cite{mouloua_2001}. 

Considering that Combat Service Support is strong related to delivery systems, the use of autonomous drones can be promising. Drone delivery is one of the emerging areas of drone applications. These types of applications are very popular today because large companies have adopted this delivery system. For instance, the automation of Amazon package delivery enhances customer service by being able to provide rapid package delivery. UPS, Google and other European delivery services are also experimenting with delivery drones \cite{mack_2018}.


\section{Unmanned Aerial Vehicles}

The U.S. Department of Defense \cite{staff_2001} defines an unmanned aerial vehicle as a powered aerial vehicle that does not carry a human operator, uses aerodynamic forces to provide vehicle lift, can fly autonomously or be piloted remotely, can be expendable or recoverable, and can carry a lethal or non-lethal payload. UAVs can perform missions unconstrained by shift schedules or human endurance, conducting more surveillance and collecting more information than humans. Moreover, UAVs can execute a targeted strike with precision \cite{foust_2012}. 

UAVs have been used successfully for military purposes -- mainly reconnaissance -- since the 1950s \cite{sullivan_2005}. The Iraq war shifted the usage from strict reconnaissance to key weapon system performing many roles that are central to the operations \cite{fahlstrom_2012}. Currently, UAVs are essential for diverse military applications and can be categorized into fixed-wing and rotatory-wing (e.g. helicopters). Both categories are often used in military operations. Some UAVs weigh hundreds or even thousands of pounds and can fly more than 6000 feet while others, named small or micro UAVs, weigh less than 10 pounds and fly usually under 1,000 feet \cite{chao_2007}.

UAVs can be classified into groups based on its size, Maximum Gross Takeoff Weight (MGTW), Normal Operating Altitude (NOA) and Airspeed \cite{dempsey_2010}. Table \ref{tab:CategoryOfUAVs} shows the classification of UAVs based on these criteria. Notice that if an UAV has even one characteristic of the next level, it is classified in that level.

\begin{table}[hbt]
\centering
\caption{Categories of UAVs. Adapted from \cite{dempsey_2010}.}
\label{tab:CategoryOfUAVs}
\begin{tabular}{lllll}
\hline
Category & Size & MGTW (lbs) & NOA (ft) & Airspeed (knots) \\
\hline
Group 1  & Small   & 0-20                               & \textless1,200 Above Ground Level & \textless100     \\
Group 2  & Medium  & 21-55                              & \textless3,500                    & \textless250     \\
Group 3  & Large   & \textless1320                      & \textless18,000 Mean Sea Level    & \textless250     \\
Group 4  & Larger  & \textgreater1320                   & \textless18,000 Mean Sea Level    & Any airspeed     \\
Group 5  & Largest & \textgreater1320                   & \textgreater18,000                & Any airspeed  \\  
\hline
\end{tabular}
\end{table}

The majority of UAVs are designed for intelligence, surveillance and reconnaissance purposes. However, some UAVs are larger and armed and can more accurately be described as unmanned combat aerial vehicles (UCAVs) \cite{mayer_2015}. According to Wang \cite{wang_2012}, this is one of the inevitable trends of the modern aerial weapon equipment owing to its potential to perform dangerous, repetitive tasks in remote and hazardous environments. Research on UCAV directly affects battle effectiveness of the air force and is a fundamental field of research for the safeness of a nation \cite{wang_2012}. 

There are two classes of UAVs: non-autonomous and autonomous. Non-autonomous UAVs are remotely controlled by humans, and autonomous are capable of processing higher level intent and direction. From this understanding and its perception of its environment, such a system is able to take appropriate actions to bring about a desired state, and is capable of deciding a course of action, from a number of alternatives faster than non-autonomous counterparts, without depending on human oversight and control \cite{mod_2011, bellamy_2015}.

\section{Vehicle Routing problem}

Vehicle Routing Problem was first proposed by Dantzig and Ramsar 1959, and is mainly used to solve the transport route optimization problem: Atlanta's refinery problem. The subject quickly attracted the attention of experts and scholars, such as operations research, management, computer, graph theory, and proved to be widely used in transportation system, logistics distribution system and express delivery system. After several decades of development, the vehicle routing problem has become an important part of logistics management research, and is classified as the general term for such a type of problem: by a number of vehicles from one or more warehouses to multiple geographic On the distribution of customers, how to arrange the vehicle and its route to the total distribution costs can be minimized. 

Then, in theory, the VRP problem is defined as: organizing a series of loading and unloading points, as well as the corresponding traffic line organization, so that vehicles can be ordered through them. That is, to achieve the objectives and solve certain problems (such as shortest distance, minimum cost, time limit) under certain constraints (goods demand, delivery, delivery time,
vehicle capacity constraints, travel restrictions, time constraints, etc.)

VRP problems have a variety of classification methods and, for different classification methods, also have different corresponding models and algorithms. However, no matter how complex the model of the algorithm, common constraints in the discussion of VRP problem is the same capacity constraints [3]:

\begin{itemize}
    \item \textbf{Capacity constraints:} regardless of which vehicle, it should be less than the total path of the vehicle load capacity. 
    \item \textbf{Priority constraints:} leads to the priority of the vehicle to constrain the path problem.
    \item \textbf{Vehicle constraints:} leads to multi-vehicle vehicle routing problem.
    \item \textbf{Time window constraints:} including hard time window and soft time window constraints. The vehicle routing problem with time window.
    \item \textbf{Compatibility constraints:} leads to compatibility constraints of the vehicle routing problem.
    \item \textbf{Random demand:} leads to a random demand for the vehicle routing problem.
    \item \textbf{Multi-transport center:} leads to a multi-transport vehicle routing problem.
    \item \textbf{Return transport:} leads to a return route with the vehicle routing problem.
    \item \textbf{Vehicle speed changes with time:} with time, leads to the vehicle speed with time changes in vehicle routing problem
\end{itemize}

\subsection{Static Vehicle Routing problem}

\subsection{Dynamic Vehicle Routing problem}

The dynamic vehicle routing problem calls for online algorithms that work in real-time since the immediate requests should be served, if possible. As conventional static vehicle routing problems are NP hard, it is not always possible to find optimal solutions to problems of realistic sizes in a reasonable amount of computation time. This implies that the dynamic vehicle routing problem also belongs to the class of NP hard problems,
since a static VRP should be solved each time a new immediate request is received.























%Recent research explored the benefits of using multiple UAVs to better respond to hostile environments with active adversaries \cite{kim_2012, bekmezci_2013, bellingham_2002}. For instance, the U.S. developed a swarm of UAVs for surveillance purposes \cite{friedman_2010}. The goal of this application is to create and maintain a current picture of all activity in a battle zone. This picture is useful for targeting using navigationally guided weapons. In the context of a swarm of UAVs, the communication between the UAVs makes it possible for the swarm as an entity to decide which UAVs are best suited to engage a given target, given factors such as their position, their fuel state, and what weapons they have on board.

%Two approaches can be taken when using UAVs on missions on hostile environments. It is possible to use cheap, simple and expendable UAVs and, if any of them is destroyed, the loss is not so significant. It is also possible to use UAVs that are more complex, and therefore expensive, but most likely to survive. These UAVs would probably involve technologies aimed at increasing their physical stealth and would possess defensive aid suites \cite{mod_2011}. However, according to Kacena \cite{kacena1995}, stealth is not only restricted to usage of technologies, but can also be obtained through techniques -- cooperative or not. 

%\section{Stealth}

%Over the past few decades, stealth technology has proven to be one of the most effective approaches to hiding from radar systems \cite{zikidis_2014}. However, this technology is normally expensive, and thus the economic trade-off may not be suitable for all operations. Moreover, stealth technology can improve, but not ensure stealth along the flight. Stealth is not, however, restricted to the use of sophisticated technology: it can also be obtained though techniques which allow surprising the adversaries. These techniques do not only increase the effectiveness, of weapon systems but also increase their efficiency and combat capabilities by shrouding them with a true sense of stealth and deception to hide in plain sight. Naturally, stealth technologies and techniques can be simultaneously used \cite{fisk_2015}. 

%In terms of techniques, it is possible to reduce the airplane signature with simple approaches, such as using natural hiding places provided by the environment, such as clouds and darkness \cite{haffa_2002}. In addition, radio or radar signals transmitted by an aircraft can also be used to detect an aircraft's position, therefore, techniques to decrease the probability of intercepting those signals, such as flying with radars turned off, using electronic pulses that do not exceed the range of the threat, and using frequency-hopping radios and radars to deny electromagnetic detection are all possible stealth mechanisms\cite{haffa_2002, friedman_2010}.

%Aircrafts can be equipped with different sensors and radars. Regarding stealth, an active approach to detect a threat presence can expose aircrafts to passive enemy radars. Moreover, it is highly likely that an aircraft radar will have a detection range larger than an enemy vigilance radar, therefore it would be detected in advance. Due to this situation, the detection of the threats occurs passively on hostile environments. One equipment applied for this purpose is the Radar Warning Receiver (RWR), designed to detect radars as part of the process of protecting the target against those radars. It must detect a wide range of radar signals, which can come from any direction \cite{adamy2004}. The RWR has been used successfully by modern combat aircraft, such as the F-16, for many years \cite{brownlow_2008}. Regarding UAVs, tests using RWR on Predator B/MQ-9 Reaper Block 5 were successfully conducted on April 2017, allowing UAVs to operate in the proximity of threat radars and enemy air defenses \cite{osborn_2017}. 
  
%Another technique to improve stealth is to use a collaborative approach between the nodes of the network in order to reduce the network's overall exposure to threats. This kind of technique is described in Turgut \cite{turgut_2009}, which presents a way to quantify the stealth level of a sensor node with a numerical metric and propose a local model, based on a try and bounce (TAB) approach. In this approach the nodes that are exposed, i.e., whose have a stealth level smaller than a threshold, do not reply to messages sent by other sensors, meaning that their are in a threat's region.









